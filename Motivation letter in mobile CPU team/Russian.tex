\documentclass[11pt,a4paper,roman]{moderncv}
\usepackage[russian]{babel}

\moderncvstyle{classic}
\moderncvcolor{green}

\usepackage[utf8]{inputenc}
\usepackage[colorlinks=true, urlcolor=blue, unicode]{hyperref}
\usepackage[scale=0.75]{geometry}

\name{Соложенкин}{Севастьян Алексеевич}

\phone[mobile]{+7(903)729-59-70}
\email{sevsol2512@gmail.com}
\date{}

\begin{document}

\opening{Здравствуйте!}
\closing{С уважением, }

\makelettertitle
\hspace{0.5cm}
Меня зовут Соложенкин Севастьян, я обучаюсь на первом курсе ФРКТ МФТИ по направлению "Информатика и вычислительная техника".

\hspace{0.5cm}
В школе я не знал, чем конкретно хочу заниматься. Мне очень нравилось учить физику и математику, а также программировать.
Я проходил различные курсы по программированию, начиная от курсов Сириуса и заканчивая Яндекс Лицеем.
В момент поступления в вуз мне было очень сложно определиться с местом дальнейшего обучения.
Я выиграл множество олимпиад по физике и математике, что открывало мне путь практически в любой университет на любое направление.
Я выбрал МФТИ, так как в этом вузе очень сильное окружение и много возможностей.
Выбор физтех-школы пал на ФРКТ, так как здесь в одинаковой мере можно углубиться и в математику, и в физику, и в программирование.
Я хотел, чтобы у меня было время выбрать, чему именно я посвящу свою жизнь.

\hspace{0.5cm}
Попав на летнюю школу программирования Ильи Рудольфовича Дединского, я понял, что мой выбор — программирование.
Курс Дединского сочетает в себе интересные большие проекты, плавно переходящие друг в друга.
Я занимался программированием почти всё время.

\hspace{0.5cm}
Сейчас мне надо выбрать, какой областью IT я буду заниматься.
Выбор не менее тяжёлый, чем тот, который стоял передо мной по окончании школы.

\hspace{0.5cm}
Мне понравилось оптимизировать программы и анализировать сгенерированный компилятором код \href{https://github.com/sevaphasol/mandelbrot}{в задании из курса И.Р. Дединского}.
Поэтому из предложенных направлений меня больше всего заинтересовала группа мобильного ЦП под руководством Сергея Лисицына. Я прочитал \href{https://cyberleninka.ru/article/n/sbor-profilnoy-informatsii-s-pomoschyu-trass-ispolneniya-prilozheniya-dlya-staticheskoy-optimiziruyuschey-binarnoy-translyatsii/viewer}{статью} об оптимизации исполняемого кода перестановкой "горячих"и "холодных"функций. Я думаю, что мне было бы интересно развиваться в данном направлении. Поэтому был бы рад возможности поучаствовать в летней стажировке с целью получения опыта.

\makeletterclosing

\end{document}
